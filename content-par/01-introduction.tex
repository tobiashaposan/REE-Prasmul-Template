\subsection{Overview}
Recent studies of perovskite solar cells (PSCs) demonstrate that PSCs have higher experimental power conversion efficiency (PCE) when compared to silicon-based solar cells, with a record efficiency of 25.7\% as of November 2022 \cite{sharma_stability_2022, zhang_review_2022, basumatary_short_2022}. In manufacturing terms, the simplicity of large-scale fabrication such as ink-jet printing coupled with room-temperature precursor preparation conditions makes PSCs suitable for low-cost manufacturing \cite{rong_toward_2018, li_cost_2018, ahangharnejhad_impact_2022}. Unlike silicon-based solar cells, organic-inorganic PSCs can be made using solution-based precursor material. In manufacturing terms, the simplicity of utilizing solution-based precursor material for large-scale fabrication such as ink-jet printing coupled with room-temperature precursor preparation conditions makes PSCs suitable for low-cost manufacturing \cite{rong_toward_2018, li_cost_2018, ahangharnejhad_impact_2022}. The light-harvesting layer of PSCs is made from crystal perovskite with the structure of ABX\textsubscript{3}, where A is usually an organic cation, B is a metal cation, and X is a halide anion \cite{mahmud_origin_2022, correa-baena_promises_2017}. A light-harvesting layer can be prepared in the form of a liquified crystal solution. The solution can be deposited onto a substrate before being heated to form a solid thin film \cite{seok_methodologies_2018}. Solution deposition varies in techniques, some are suitable according to the cell area and material viscosity \cite{rong_toward_2018}. \par
The stability of PSCs is limited by rapid PCE degradation \cite{sharma_stability_2022, mahmud_origin_2022, seok_methodologies_2018, pean_investigating_2020, chen_synergistic_2019}. Perovskite solar cells (PSCs) are known to be unstable, which is a major challenge in developing and commercialising these devices. One source of instability is the tendency of the inorganic substances in PSCs to form hydrated products when exposed to moisture \cite{correa-baena_promises_2017, chen_synergistic_2019, cho_mixed_2018}. This can lead to rapid degradation of the power conversion efficiency (PCE) of the solar cell. One key idea for addressing this instability is to incorporate a durable organic material into the PSC to help protect it from moisture and other environmental factors. By doing so, it may be possible to improve the stability and long-term performance of PSCs \cite{seok_methodologies_2018, maafa_all-inorganic_2022}. \par
Dimensional engineering of perovskite thin film shows remarkable enhancement in both PCE and stability \cite{mahmud_origin_2022}. 2D perovskite materials are characterized by the presence of bulky organic cations, which impart several interesting properties to these materials. One property of bulky organic cations is that they are hydrophobic, which means that they do not quickly form a hydrated product when exposed to moisture, caused by the property of bulky organic cations is that they have a high barrier to ion migration, which can improve the stability of the material \cite{chen_phase_2018}. 3D perovskite materials were actually the first form of hybrid perovskite solar cells (PSCs) to be developed. One key property of 3D perovskites is that they have a lower bandgap than 2D perovskites, which means that they can absorb a wider range of wavelengths and can potentially be more efficient as light-absorbing materials. However, 3D perovskites are generally less durable than 2D perovskites when exposed to moisture and other environmental factors. In 2014, a 2D/ 3D combination showed that the respective perovskite has a slower scope of degradation and improved efficiency when combining both structures, showing it is possible to mitigate the effects of the higher bandgap by mixing 2D perovskite with conventional 3D perovskite \cite{smith_layered_2014}. According to the study, it is possible to combine both 2D and 3D perovskite materials in order to retain the desirable properties of both while omitting their weaknesses. This approach was shown to lead to an increase in power conversion efficiency (PCE) from around 2\% for 2D perovskite to 4.73\% for the hybrid material. This approach can help to improve the performance of 2D perovskite solar cells. Among the bulky organic cations, FA, MA, and PEA can be prepared and synthesized at room temperature with ambient air with relatively high stability \cite{krishna_mixed_2019, chen_stabilizing_2017, grancini_one-year_2017, cao_2d_2015, mesquita_effect_2020}. An experiment conducted by Lee et al. shows that a configuration of 3D/2D FA-MA/PEA with carbon electrode shows the highest efficiency of 14.9\% and degradation to 13.7\% after 1000 hours of operation. This experiment will attempt to replicate such a device and investigate the effect each organic has on the PSC performance.
\subsection{Problem formulation}
This experiment aims to investigate how organics affect the performance of PSC. Therefore, the goal of this experiment is to create a multidimensional hybrid perovskite solar cell and investigate the effect of the organics on performance by replicating the experiment done by Lee \textit{et al.} \cite{lee_highly_2018}. 
\subsection{Objectives}
The objective of this experiment is to fabricate hybrid 2D/ 3D perovskite solar cells (PSCs) using p-ethylenediamine (PEA), methylamine (MA), and formamidinium (FA) as organics. The resulting 2D/ 3D layers will be characterized using scanning electron microscopy and energy dispersive x-ray spectroscopy (SEM-EDX). Cell performance data will be extracted using a solar simulator, and an interaction plot will be generated to study the effect of the organics on the performance of the hybrid 2D/ 3D PSCs. This information will help to improve our understanding of the role that these organics play in the performance of hybrid 2D/ 3D PSCs and could potentially lead to the development of more efficient solar cells in the future.
\subsection{Experiment benefits}
The PSCs fabricated would then be characterized to determine their performance and efficiency, involving the measurements of their electrical properties, such as their current-voltage characteristics, and assessing their ability to convert light into electricity. The goal of the experiment would be to gain a better understanding of the organics in hybrid perovskite materials in solar cell technology and to identify any potential challenges or limitations in their use.
\subsection{Scope of experiment}
The scope of this experiment is to investigate the use of DMF:DMSO and chlorobenzene as solvents for the preparation of perovskite light harvester precursor materials, and to use PEA, MA, and FA as organic components in the fabrication of hybrid 2D/ 3D perovskite solar cells. MA and FA will be coupled to form 3D perovskite, while PEA will be used for 2D perovskite only. All annealing and sintering processes will be carried out at room temperature, and TiO\textsubscript{2} and carbon will be used as the electron transport layer (ETL) and back contact, respectively. There are several limitations to this experiment. One limitation is that only DMF:DMSO and chlorobenzene will be used as solvents for the preparation of the perovskite light harvester precursor materials, so the results may not be applicable to other solvents. Additionally, the use of only PEA, MA, and FA as organic components may not fully capture the range of possible organic compositions that could be used in the fabrication of hybrid 2D/ 3D perovskite solar cells. Finally, the fact that all annealing and sintering processes are carried out at room temperature may impact the quality and performance of the resulting solar cells.